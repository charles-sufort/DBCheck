\section{Conclusion}
The results of DBCheck show the potential of this form of Model Checking in a tool for model validation. The stated properties of the 2PL protocols variants are correspondingly observed in the results of the specification testing. The results, however, speak for the model tested and the 2PL protocols themselves. Further efforts could be underdone to formalize the protocol standard and then find a representative model from which general results. A verfication fo the Kerberos protocol is done with NuSMV by M. Panti L. Spalazzi and S. Tacconi \cite{kerberos}. 

Since the 2PL standards themselves follow a loose system of definition, another application is in the validation of the implementation model of the protocol. From this vantage, the models instead are supposed to represent specific architectures using the 2PL standard. In the DBCheck implementation, the models are abstract and miniature in scale. Since the validation of the model corresponds only so far as the model represents the behavior of the system, a further objective here is creating models that better represent the system behavior. With more exhaustive model definition, the opportunites for the verification of ACID properties such as durability also will open up. For example, this could be seen in the DBCheck models if the models also accounted for occurences such as abort sequences and system failures.

One barrier to the use of model checking are the complexity challenges involved. An example of this is seen in DBCheck as the number of possible schedules. For the models in use, the runtime of any model with a specific schedule is minimal (seconds or less). The problem arrises however in the combinatorics of the possible schedules that the system may encounter. This value is exponentially correlated to the number of operations in a schedule. As a result, the model checking may prove infeasible with schedules of arbitrary size. 

In conclusion, the applicability of NuSMV for the verification purposes of both implemented system and protocol is clearly apperent. The use of model checking for both purposes is the site of substantial application. Further research in the database concurrency and transaction management could involve more rigorous specfication of ACID properties within the domain of temporal logic. The computational models of database system could be further defined for the comparison of results. A potential outcome of this research would be a the design of an tool and accompanying engineering testing methodology for the specfication and testing of DBMS systems.

